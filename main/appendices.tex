\appendix 
\section{Transient analysis} 
\label{app:transient_analysis}
This section is as a convenient check for myself, just as the table of the next section
\begin{figure}[ht!]
	\centering 
	\includegraphics[width=\linewidth]{Code/Figures/closed_loop_tran.pdf}
	\caption{Ordinary transient analysis for the step input}
\end{figure}
\subsection{Final values}
\label{sec:final_values}

\begin{equation}
    \begin{cases} 
        S_a = 0.986 \\ 
        R_{34} = 95.0 \\ 
        R_{mp} = 0.907 \\ 
        I_{bmain} = 572.1 \mu A \\ 
        C_{in} = 32.36 p F \\ 
        S_{ab} = 16.907
    \end{cases}
\end{equation}

\newpage
\begin{landscape}
    \section{Operating points}
	\begin{table}[ht!]
		\centering 
		\begin{tabular}{lllllllllll}
\toprule
 & m:x1:n1 & m:x1:n2 & m:x1:n3 & m:x1:n4 & m:x1:n5 & m:x1:n6 & m:x1:n7 & m:x1:n8 & m:x1:p1 & m:x1:p2 \\
\midrule
$\vert V_{Th} \vert$ & 0.53 & 0.57 & 0.63 & 0.63 & 0.53 & 0.57 & 0.53 & 0.57 & 0.49 & 0.63 \\
$\vert V_{GS} \vert$ & 0.63 & 0.66 & 0.49 & 0.49 & 0.63 & 0.65 & 0.63 & 0.65 & 0.82 & 0.96 \\
$\vert V_{DS}+V_{Th} \vert$ & 0.67 & 0.83 & 1.58 & 1.58 & 0.68 & 1.31 & 0.68 & 1.31 & 0.94 & 1.09 \\
$\vert V_{GT}-V_{Th} \vert$ & 0.11 & 0.10 & -0.14 & -0.14 & 0.11 & 0.08 & 0.11 & 0.08 & 0.32 & 0.33 \\
Saturation? & True & True & False & False & True & True & True & True & True & True \\
Model: & x1:nch.9 & x1:nch.9 & x1:nch.9 & x1:nch.9 & x1:nch.9 & x1:nch.9 & x1:nch.9 & x1:nch.9 & x1:pch.9 & x1:pch.9 \\
Id: & 65.90 & 65.90 & 33.00 & 33.00 & 33.90 & 33.90 & 33.90 & 33.90 & -66.80 & -33.90 \\
Vgs: & 6.33e-01 & 6.62e-01 & 4.93e-01 & 4.93e-01 & 6.34e-01 & 6.50e-01 & 6.34e-01 & 6.50e-01 & -8.17e-01 & -9.58e-01 \\
Vds: & 1.43e-01 & 2.64e-01 & 9.50e-01 & 9.50e-01 & 1.55e-01 & 7.46e-01 & 1.55e-01 & 7.46e-01 & -4.43e-01 & -4.57e-01 \\
Vbs: & 0.00e+00 & -1.43e-01 & -4.07e-01 & -4.07e-01 & 0.00e+00 & -1.55e-01 & 0.00e+00 & -1.55e-01 & 0.00e+00 & 4.43e-01 \\
Vth: & 5.26e-01 & 5.67e-01 & 6.30e-01 & 6.30e-01 & 5.26e-01 & 5.69e-01 & 5.26e-01 & 5.69e-01 & -4.94e-01 & -6.30e-01 \\
Vdsat: & 1.17e-01 & 1.12e-01 & 4.61e-02 & 4.61e-02 & 1.17e-01 & 1.06e-01 & 1.17e-01 & 1.06e-01 & -2.65e-01 & -2.80e-01 \\
Gm: & 6.87e-04 & 7.56e-04 & 7.79e-04 & 7.79e-04 & 3.55e-04 & 4.06e-04 & 3.55e-04 & 4.06e-04 & 3.35e-04 & 1.67e-04 \\
Gds: & 1.25e-04 & 4.65e-05 & 1.70e-05 & 1.70e-05 & 5.48e-05 & 1.03e-05 & 5.48e-05 & 1.03e-05 & 2.86e-05 & 1.46e-05 \\
Gmb & 2.05e-04 & 2.04e-04 & 1.79e-04 & 1.79e-04 & 1.06e-04 & 1.07e-04 & 1.06e-04 & 1.07e-04 & 1.15e-04 & 4.85e-05 \\
Cbd: & 0.00e+00 & 0.00e+00 & 0.00e+00 & 0.00e+00 & 0.00e+00 & 0.00e+00 & 0.00e+00 & 0.00e+00 & 0.00e+00 & 0.00e+00 \\
Cbs: & 0.00e+00 & 0.00e+00 & 0.00e+00 & 0.00e+00 & 0.00e+00 & 0.00e+00 & 0.00e+00 & 0.00e+00 & 0.00e+00 & 0.00e+00 \\
Cgsov: & 1.31e-15 & 1.31e-15 & 3.12e-14 & 3.12e-14 & 6.57e-16 & 6.57e-16 & 6.57e-16 & 6.57e-16 & 1.29e-15 & 6.46e-16 \\
Cgdov: & 1.31e-15 & 1.31e-15 & 3.12e-14 & 3.12e-14 & 6.57e-16 & 6.57e-16 & 6.57e-16 & 6.57e-16 & 1.29e-15 & 6.46e-16 \\
Cgbov: & 1.26e-16 & 1.26e-16 & 2.98e-15 & 2.98e-15 & 6.28e-17 & 6.28e-17 & 6.28e-17 & 6.28e-17 & 1.66e-16 & 8.32e-17 \\
dQgdVgb: & 7.04e-15 & 6.91e-15 & 9.89e-14 & 9.89e-14 & 3.52e-15 & 3.40e-15 & 3.52e-15 & 3.40e-15 & 6.79e-15 & 3.38e-15 \\
dQgdVdb: & -1.37e-15 & -1.31e-15 & -3.12e-14 & -3.12e-14 & -6.79e-16 & -6.52e-16 & -6.79e-16 & -6.52e-16 & -1.29e-15 & -6.42e-16 \\
dQgdVsb: & -4.90e-15 & -4.80e-15 & -3.27e-14 & -3.27e-14 & -2.45e-15 & -2.33e-15 & -2.45e-15 & -2.33e-15 & -5.14e-15 & -2.58e-15 \\
dQddVgb: & -1.50e-15 & -1.35e-15 & -3.12e-14 & -3.12e-14 & -7.34e-16 & -6.59e-16 & -7.34e-16 & -6.59e-16 & -1.32e-15 & -6.60e-16 \\
dQddVdb: & 1.50e-15 & 1.34e-15 & 3.12e-14 & 3.12e-14 & 7.30e-16 & 6.57e-16 & 7.30e-16 & 6.57e-16 & 1.32e-15 & 6.60e-16 \\
dQddVsb: & 5.89e-17 & 2.80e-17 & 1.04e-18 & 1.04e-18 & 2.83e-17 & 2.04e-18 & 2.83e-17 & 2.04e-18 & 1.02e-17 & 4.31e-18 \\
dQbdVgb: & -9.54e-16 & -9.86e-16 & -3.50e-14 & -3.50e-14 & -4.81e-16 & -5.03e-16 & -4.81e-16 & -5.03e-16 & -1.00e-15 & -4.81e-16 \\
dQbdVdb: & -5.59e-17 & -5.38e-18 & 1.22e-18 & 1.22e-18 & -2.23e-17 & 8.95e-19 & -2.23e-17 & 8.95e-19 & -8.58e-18 & -3.96e-18 \\
dQbdVsb: & -8.12e-16 & -7.06e-16 & -2.55e-16 & -2.55e-16 & -4.06e-16 & -3.40e-16 & -4.06e-16 & -3.40e-16 & -4.35e-16 & -1.26e-16 \\
\bottomrule
\end{tabular}

		\caption{Voltages from the second operating point simulation}
		\label{tab:operating_point1}
	\end{table}
	Transistors $n3$ and $n4$ are in subthreshold since this is a more efficient region to operate in.
	\begin{table}[ht!]
		\centering 
		\begin{tabular}{lllllllll}
\toprule
 & m:x1:p3 & m:x1:p4 & m:x1:x1:n1 & m:x1:x1:n2 & m:x1:x1:n3 & m:x1:x1:p1 & m:x1:x1:p2 & m:x1:x1:p3 \\
\midrule
$\vert V_{Th} \vert$ & 0.49 & 0.63 & 0.53 & 0.57 & 0.53 & 0.49 & 0.63 & 0.49 \\
$\vert V_{GS} \vert$ & 0.82 & 0.96 & 0.63 & 0.66 & 0.81 & 0.82 & 0.96 & 1.40 \\
$\vert V_{DS}+V_{Th} \vert$ & 0.94 & 1.09 & 0.68 & 1.05 & 1.33 & 0.93 & 1.01 & 1.89 \\
$\vert V_{GT}-V_{Th} \vert$ & 0.32 & 0.33 & 0.11 & 0.09 & 0.28 & 0.32 & 0.33 & 0.91 \\
Saturation? & True & True & True & True & True & True & True & True \\
Model: & x1:pch.9 & x1:pch.9 & x1:x1:nch.9 & x1:x1:nch.9 & x1:x1:nch.9 & x1:x1:pch.9 & x1:x1:pch.9 & x1:x1:pch.9 \\
Id: & -66.80 & -33.90 & 572.00 & 572.00 & 2750.00 & -572.00 & -572.00 & -3550.00 \\
Vgs: & -8.17e-01 & -9.58e-01 & 6.33e-01 & 6.55e-01 & 8.05e-01 & -8.17e-01 & -9.62e-01 & -1.40e+00 \\
Vds: & -4.43e-01 & -4.57e-01 & 1.50e-01 & 4.83e-01 & 8.05e-01 & -4.39e-01 & -3.78e-01 & -1.40e+00 \\
Vbs: & 0.00e+00 & 4.43e-01 & 0.00e+00 & -1.50e-01 & 0.00e+00 & 0.00e+00 & 4.39e-01 & 0.00e+00 \\
Vth: & -4.94e-01 & -6.30e-01 & 5.26e-01 & 5.68e-01 & 5.25e-01 & -4.94e-01 & -6.29e-01 & -4.90e-01 \\
Vdsat: & -2.65e-01 & -2.80e-01 & 1.17e-01 & 1.09e-01 & 1.97e-01 & -2.65e-01 & -2.83e-01 & -5.80e-01 \\
Gm: & 3.35e-04 & 1.67e-04 & 5.99e-03 & 6.79e-03 & 1.35e-02 & 2.87e-03 & 2.72e-03 & 5.80e-03 \\
Gds: & 2.86e-05 & 1.46e-05 & 9.85e-04 & 2.25e-04 & 4.88e-04 & 2.48e-04 & 3.41e-04 & 3.59e-04 \\
Gmb & 1.15e-04 & 4.85e-05 & 1.79e-03 & 1.82e-03 & 3.94e-03 & 9.86e-04 & 7.96e-04 & 2.03e-03 \\
Cbd: & 0.00e+00 & 0.00e+00 & 0.00e+00 & 0.00e+00 & 0.00e+00 & 0.00e+00 & 0.00e+00 & 0.00e+00 \\
Cbs: & 0.00e+00 & 0.00e+00 & 0.00e+00 & 0.00e+00 & 0.00e+00 & 0.00e+00 & 0.00e+00 & 0.00e+00 \\
Cgsov: & 1.29e-15 & 6.46e-16 & 1.13e-14 & 1.13e-14 & 1.13e-14 & 1.11e-14 & 1.11e-14 & 1.11e-14 \\
Cgdov: & 1.29e-15 & 6.46e-16 & 1.13e-14 & 1.13e-14 & 1.13e-14 & 1.11e-14 & 1.11e-14 & 1.11e-14 \\
Cgbov: & 1.66e-16 & 8.32e-17 & 1.08e-15 & 1.08e-15 & 1.08e-15 & 1.43e-15 & 1.43e-15 & 1.43e-15 \\
dQgdVgb: & 6.79e-15 & 3.38e-15 & 6.03e-14 & 5.87e-14 & 6.25e-14 & 5.82e-14 & 5.81e-14 & 5.97e-14 \\
dQgdVdb: & -1.29e-15 & -6.42e-16 & -1.17e-14 & -1.12e-14 & -1.12e-14 & -1.11e-14 & -1.11e-14 & -1.10e-14 \\
dQgdVsb: & -5.14e-15 & -2.58e-15 & -4.20e-14 & -4.05e-14 & -4.52e-14 & -4.41e-14 & -4.42e-14 & -4.64e-14 \\
dQddVgb: & -1.32e-15 & -6.60e-16 & -1.27e-14 & -1.14e-14 & -1.13e-14 & -1.13e-14 & -1.15e-14 & -1.11e-14 \\
dQddVdb: & 1.32e-15 & 6.60e-16 & 1.26e-14 & 1.13e-14 & 1.13e-14 & 1.13e-14 & 1.16e-14 & 1.11e-14 \\
dQddVsb: & 1.02e-17 & 4.31e-18 & 4.95e-16 & 8.12e-17 & 3.92e-17 & 8.84e-17 & 6.79e-17 & 1.32e-17 \\
dQbdVgb: & -1.00e-15 & -4.81e-16 & -8.22e-15 & -8.58e-15 & -8.69e-15 & -8.57e-15 & -8.19e-15 & -8.90e-15 \\
dQbdVdb: & -8.58e-18 & -3.96e-18 & -4.18e-16 & 8.57e-18 & 4.00e-18 & -7.64e-17 & -1.56e-16 & 1.02e-18 \\
dQbdVsb: & -4.35e-16 & -1.26e-16 & -6.96e-15 & -5.91e-15 & -6.56e-15 & -3.73e-15 & -2.16e-15 & -2.78e-15 \\
\bottomrule
\end{tabular}

		\caption{Voltages from the second operating point simulation}
		\label{tab:operating_point2}
	\end{table}
\end{landscape}
\newpage
\section{Simulation}
Symbol and Python files not included
\subsection{Biasing circuit}
\lstinputlisting[breaklines]{Code/Circuits/Biasing_circuit.asc}
\subsection{Opamp}
\lstinputlisting[breaklines]{Code/Circuits/op_amp.asc}
\subsection{CMFB}
\lstinputlisting[breaklines]{Code/Circuits/CMFB.asc}
\subsection{Gain-boosting}
\lstinputlisting[breaklines]{Code/Circuits/gain_boosting_nmos.asc}
\lstinputlisting[breaklines]{Code/Circuits/gain_boosting_pmos.asc}
\subsection{Closed loop}
\lstinputlisting[breaklines]{Code/Circuits/closed_loop.asc}
